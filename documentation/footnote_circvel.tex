\paragraph{$^{$FOOTNOTE_NUMBER$}$The maximum circular velocity and the radius where it is reached}\label{footnote:$FOOTNOTE_NUMBER$} are 
computed using

\begin{equation}
    v_{\rm{}max} = \sqrt{\frac{G M(\leq{}r)}{r}},
\end{equation}

where the cumulative mass $M(\leq{}r)$ includes all particles within the radius $r$, and includes the
contribution of the particle(s) at $r=0$. The radius is computed relative to the halo centre.
The softened $v_{\rm{}max}$ value is calculated using the same method, except the particle
radius has a floor of the softening length. An alternative way to calculate $v_{\rm{}max}$
is to estimate it from the halo concentration by assuming an NFW profile. We store the radius of the
unsoftened maximum circular velocity. If the softened and unsoftened maximum circular velocities are
equal, then their radii will also be equal. If the values are not equal, then the radius of the
softened maximum circular velocity will be the simulation softening length.
