\paragraph{$^{$FOOTNOTE_NUMBER$}$Dust quantities assume a six element dust model}\label{footnote:$FOOTNOTE_NUMBER$} with one type of graphite 
grains and two types of silicate grains, each of which has two size bins. The dust mass fractions are
then based on the \verb+DustMassFractions+ dataset in the snapshot, assuming the following order of the
six columns:

\begin{enumerate}
  \item Large graphite grains
  \item Large silicate grains of type 1
  \item Large silicate grains of type 2
  \item Small graphite grains
  \item Small silicate grains of type 1
  \item Small silicate grains of type 2
\end{enumerate}

The total graphite mass, \verb+DustGraphiteMass+, is then for example calculated by summing the masses 
contained in columns 1 and 3.

The mass of dust in molecular gas is defined as
\begin{equation}
    \sum_i m_d \frac{m_{\rm{H}_2,i}}{m_{\rm{H}, i}}
\end{equation}
where $m_d$ is the dust mass of particle $i$, $m_{\rm{H}_2,i}$ is the molecular hydrogen mass of the particle,
and $m_{\rm{H},i}$ is the total hydrogen mass of the particle. A similar expression is used for the mass of dust in atomic gas.
