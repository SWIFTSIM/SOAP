\paragraph{$^{$FOOTNOTE_NUMBER$}$The Compton y parameter} is computed as in McCarthy et al. (2017):

\begin{equation}
    y \, {d_A}^2(z) = \sum_i \frac{\sigma{}_T}{m_e c^2} n_{e,i} k_B T_{e,i} V_i,
\end{equation}

where $d_A(z)$ is the angular diameter distance, $\sigma{}_T$ is the Thomson cross section, $m_e$ the electron mass, $c$ the speed of light and $k_B$ the 
Boltzmann constant. $n_{e,i}$ and $T_{e,i}$ are the electron number density and electron temperature for gas 
particle $i$, while $V_i=m_i/\rho{}_i$ is the SPH volume element that turns the sum over all particles $i$ 
within the inclusive sphere into a volume integral. Note that the snapshot already contains the individual 
$y_i$ values for the SPH particles, computed from the cooling tables during the simulation.
