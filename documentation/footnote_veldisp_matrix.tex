\paragraph{$^{$FOOTNOTE_NUMBER$}$The velocity dispersion matrix} is defined as

\begin{equation}
    V_{\rm{}disp,comp} = \frac{1}{\sum_{i={\rm{}comp}} m_i} \sum_{i={\rm{}comp}} m_i \vec{v}_{{\rm{}comp},r,i}\vec{v}_{{\rm{}comp},r,i},
\end{equation}

where we compute the relative velocity as before, i.e. w.r.t. the centre of mass velocity of the particular 
component of interest. While it is strictly speaking a $3\times{}3$ matrix, there are only 6 independent 
components. We use the following convention to output those 6 components as a 6 element array:

\begin{equation}
    V'_{\rm{}disp} = \begin{pmatrix}
    V_{xx} & V_{yy} & V_{zz} & V_{xy} & V_{xz} & V_{yz}
    \end{pmatrix}.
\end{equation}

Other velocity dispersion definitions can be derived from this general form. The one-dimensional velocity dispersion can be calculated as

\begin{equation}
    \sigma = \sqrt{\frac{V_{xx} + V_{yy} + V_{zz}}{3}}
\end{equation}
