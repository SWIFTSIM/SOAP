\paragraph{$^{$FOOTNOTE_NUMBER$}$The angular momentum} of gas, dark matter and stars is computed relative to 
the halo centre (cop) and the centre of mass velocity of that particular component, and not to the 
total centre of mass velocity. The full expression is

\begin{equation}
    \vec{L}_{\rm{}comp} = \sum_{i={\rm{}comp}} m_i \left(\vec{x}_{r,i} \times{} \vec{v}_{{\rm{}comp},r,i} \right),
\end{equation}

with the sum $i$ over all particles of that particular component (bound to the halo), and

\begin{equation}
    \vec{x}_{r,i} = \vec{x}_i - \vec{x}_{\rm{}cop},
\end{equation}

\begin{equation}
    \vec{v}_{{\rm{}comp},r,i} = \vec{v}_i - \vec{v}_{\rm{}com,comp},
\end{equation}

where

\begin{equation}
    \vec{v}_{\rm{}com,comp} = \frac{\sum_{i={\rm{}comp}} m_i \vec{v}_i}{\sum_{i={\rm{}comp}} m_i}.
\end{equation}

For FLAMINGO, we also compute the angular momentum for baryons, where the sum is then over both gas and star 
particles.
